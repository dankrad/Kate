
\documentclass[a4paper]{article}
\usepackage[hmargin=2cm,vmargin=2cm]{geometry}


\pagestyle{plain}

\usepackage{amssymb,amsthm,amsmath,amsfonts,longtable, comment,array, ifpdf, hyperref,url}
\usepackage{graphicx}

\title{Fast amortized Kate proofs}
\author{Dmitry Khovratovich\\Ethereum Foundation}

\begin{document}

\maketitle

\section{Definitions}

\subsection{Setup}

Let $g$ be a group $\mathbb{G}$ element, and denote $[a] = a\cdot g$ a group element where $a$ is an integer. 

Let $s$ be a secret, then a universal setup of degree $m$ consists of  $m$ elements of $\mathbb{G}$:
$$
[s], [s^2], \ldots, [s^m].
$$


\subsection{Commitment}
Let $f(X) = \sum_{0\leq i \leq m}f_i X^i$ be a polynomial of degree $m$. Then a commitment $C_f\in \mathbb{G}$ is defined as
$$
C_f = \sum_{0\leq i \leq m} f_i[s^i],
$$
being effectively an evaluation of $f$ at point $s$.

\subsection{Proof}
Note that for any $y$ we have that $(X-y)$ divides $f(X) - f(y)$. Then the proof that $f(y) = z$ is defined as
$$
\pi[f(y)=z] = C_T,
$$
where $T_y(X) = \frac{f(X)-z}{X-y}$ is a polynomial of degree $(m-1)$.

Note that a proof can be constructed using $m$ scalar multiplications in the group. The coefficients of $T$ are computed with one multiplication each:
\begin{align}
    T_y(X) &= \sum_{0\leq i \leq m-1}t_i X^i;\\
    t_{m-1} &= f_m;\\
    t_j &= f_{j-1}+y\cdot t_{j+1} .
\end{align}
Expanding on the last equation, we get
\begin{multline}
T_y(X) =f_mX^{m-1} + (f_{m-1}+yf_{m})X^{m-2} + (f_{m-2}+yf_{m-1}+y^2f_m)X^{m-3} +\\+
(f_{m-3}+yf_{m-2}+y^2f_{m-1}+y^3)X^{m-4}+\cdots +  (f_{1}+yf_{2}+y^2f_3+\cdots+y^{m-1}f_m).
\end{multline}
\section{Multiple proofs}

Let $w$ be a $2^n$-th root of unity. We show how to construct Kate proofs for  $w,w^2,w^3,\ldots, w^{2^n}=1$. 

Note that a proof for $w^k$ is
\begin{multline}
\pi[f(w^k)=z^k] = C_{T_{w^k}}  =f_m[s^{m-1}] + (f_{m-1}+w^kf_{m})[s^{m-2}] + (f_{m-2}+w^kf_{m-1}+w^{2k}f_m)[s^{m-3}] +\\+
(f_{m-3}+w^kf_{m-2}+w^{2k}f_{m-1}+w^{3k})[s^{m-4}]+\cdots +  (f_{1}+w^kf_{2}+w^{2k}f_3+\cdots+w^{(m-1)k}f_m).
\end{multline}
Regrouping the terms, we get (for $2^n\geq m$):
\begin{align}
     C_{T_{w^k}} 
     =&\left(f_m[s^{m-1}]+f_{m-1}[s^{m-2}]+f_{m-2}[s^{m-3}]+\cdots + f_2[s]+f_1\right)+\\
     &+\left(f_m[s^{m-2}]+f_{m-1}[s^{m-3}]+f_{m-2}[s^{m-4}]+\cdots + f_3[s]+f_2\right)w^k+\\
     &+\left(f_m[s^{m-3}]+f_{m-1}[s^{m-4}]+f_{m-2}[s^{m-5}]+\cdots + f_4[s]+f_3\right)w^{2k}+\\
     &+\left(f_m[s^{m-4}]+f_{m-1}[s^{m-5}]+f_{m-2}[s^{m-6}]+\cdots + f_5[s]+f_4\right)w^{3k}+\\
     &\cdots\\
     &+(f_m[s]+f_{m-1})w^{(m-2)k}+f_mw^{(m-1)k}.
\end{align}
Let for $1
\leq i \leq 2^n$ denote $$
h_i = \left(f_m[s^{m-i}]+f_{m-1}[s^{m-i-1}]+f_{m-2}[s^{m-i-2}]+\cdots + f_{i+1}[s]+f_i\right).
$$
with $h_i=0$ for $i>m$. Then
\begin{equation}\label{eq:ct}
     C_{T_{w^k}} = h_1 + h_2w^k + h_3w^{2k}+\cdots + h_mw^{(m-1)k}.
\end{equation}

Let us denote
$$
\mathbf{C}_T = [C_{T_{w^1}},C_{T_{w^2}},\ldots,C_{T_{w^{2^n}}}]
$$
and
$$
\mathbf{h} = [h_1,h_2,\ldots,h_{2^n}].
$$
Recalling that the Discrete Fourier Transform of $\mathbf{a}=[a_1,a_2,\ldots,a_{2^n}]$ is
$$
\widehat{\mathbf{a}}= [\widehat{a_1},\widehat{a_2},\ldots,\widehat{a_{2^n}}]
$$
where 
$$
\widehat{a_k} = \sum_{i}a_iw^{ki}
$$
Therefore, from Equation~\eqref{eq:ct}
\begin{equation}
 \mathbf{C}_T = \mathrm{DFT}(\mathbf{h}), 
\end{equation}
and can be computed in $O(n2^n)$ cost given $\mathbf{h}$ using FFT algorithms.

Now we demonstrate that $\mathbf{h}$ can be also computed efficiently from $\{f_i\}$. Indeed, by definition
$$
\begin{bmatrix}
h_1\\
h_2\\
h_3\\
\cdots\\
h_{m-1}\\
h_m
\end{bmatrix}= \begin{bmatrix}
f_m &f_{m-1}&f_{m-2}&f_{m-3}&\cdots & f_1\\
0& f_m &f_{m-1}&f_{m-2}&\cdots & f_2\\
0 & 0& f_m &f_{m-1}&\cdots & f_3\\
&&\cdots&&&\\
0 & 0& 0 &0&\cdots & f_{m-1}\\
0 & 0& 0 &0&\cdots & f_m\\
\end{bmatrix}\cdot\begin{bmatrix}
[s^{m-1}]\\
[s^{m-2}]\\
[s^{m-3}]\\
\cdots\\
[s]\\
1
\end{bmatrix}
$$
The matrix 
$$
A = \begin{bmatrix}
f_m &f_{m-1}&f_{m-2}&f_{m-3}&\cdots & f_1\\
0& f_m &f_{m-1}&f_{m-2}&\cdots & f_2\\
0 & 0& f_m &f_{m-1}&\cdots & f_3\\
&&\cdots&&&\\
0 & 0& 0 &0&\cdots & f_{m-1}\\
0 & 0& 0 &0&\cdots & f_m\\
\end{bmatrix}
$$
is a \emph{Toeplitz} matrix. It is known that a multiplication of a vector by a $m\times m$ Toeplitz matrix costs $O(m\log m)$ operations using FFT \url{http://www.netlib.org/utk/people/JackDongarra/etemplates/node384.html}. 

Therefore, we can first compute $\mathbf{h}$ from SRS and then all  $2^n$ Kate proofs, i.e. $\mathbf{C}_T$, using $O(n2^n)$ multiplications compared to $O(2^{2n})$ for the naive approach (assuming $m = O(2^n)$).
\end{document} 
